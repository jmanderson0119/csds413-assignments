\section{Anime Dataset}

{\bf Attributes:} Rank, Name, Japanese\_name, Type, Episodes, Studio, Release\_season, Tags, Rating, Release\_year, End\_year, Description, Content\_Warning, Related\_Mange, Related\_anime, Voice\_actors, staff\\

{\bf Scenario:}  A streaming service is considering expanding its short anime series catalog ($<$ 25 episodes) and wants to understand how viewer ratings differ between anime TV series and movies released after 2015. The goal is to determine which format generally receives better audience reception to inform licensing and promotion strategies.\\

{\bf Research Question:} How do audience ratings compare between anime TV series and movies released after 2015, and which format generally receives higher ratings?

\subsection{Part A: Data Cleaning and Preprocessing}
First, filter your dataset so that only the variables critical for your analysis remain. Then clean your data so that there is consistency in variable types, capitalization, and handle any missing or invalid values.

\subsection{Part B: Generate Three Visualizations}
Produce the following types of plots:
\begin{itemize}
    \item \textbf{Error Bar Plot:} Show the mean and variability (e.g., standard error or 95\% confidence intervals) of the numerical variable across each category.
    \item \textbf{Barcode Chart:} Also known as a strip plot or rug plot. Shows individual data points across categories.
    \item \textbf{Histogram:} Plot the distribution of the numerical variable, grouped by the categorical variable (using hue or facet).
\end{itemize}

\subsection{Part C: Evaluate and Justify Visualization}
For the dataset:
\begin{itemize}
    \item Discuss the advantages and disadvantages of each visualization type.
    \item Decide which visualization is best for the research question.
    \item Support your answer with evidence from the plots and reasoning based on dataset size, shape, or structure.
\end{itemize}
