\section{Algorithm Performance Dataset}

{\bf Attributes:} Algorithm, Epoch, Accuracy, Trial Number\\

{\bf Scenario:} You are testing two reinforcement learning (RL) algorithms on a sequential decision task. To avoid overfitting and simulate real-world noise, you shuffle the dataset for each trial and run 10 independent trials per algorithm. For each trial, you track the accuracy across 10 training epochs (one pass through a dataset). Due to how you shuffle your data and algorithmic stochasticity, accuracy results vary across trials.\\

{\bf  Research Question:} Which algorithm performs more accurately on average across epochs, and how does the use of a visualization help you assess reliability and variation of each algorithm?

\subsection{Part A: Data Cleaning and Preprocessing}
First, filter your dataset so that only the variables critical for your analysis remain. Then clean your data so that there is consistency in variable types, capitalization, and handle any missing or invalid values.

\subsection{Part B: Generate Three Visualizations}
Produce the following types of plots:
\begin{itemize}
    \item \textbf{Error Bar Plot:} Show the mean and variability (e.g., standard error or 95\% confidence intervals) of the numerical variable across each category.
    \item \textbf{Barcode Chart:} Also known as a strip plot or rug plot. Shows individual data points across categories.
    \item \textbf{Histogram:} Plot the distribution of the numerical variable, grouped by the categorical variable (using hue or facet).
\end{itemize}

\subsection{Part C: Evaluate and Justify Visualization}
For the dataset:
\begin{itemize}
    \item Discuss the advantages and disadvantages of each visualization type.
    \item Decide which visualization is best for the research question.
    \item Support your answer with evidence from the plots and reasoning based on dataset size, shape, or structure.
\end{itemize}
