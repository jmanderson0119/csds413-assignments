\section{Best-Selling Albums Dataset}

{\bf Attributes:} Year, Ranking, Artist, Album, Genre, Worldwide Sales, Tracks, Album Length\\

{\bf Scenario:} A media analytics firm is interested in understanding whether certain genres consistently produce top-selling albums or if success is more scattered across genres.\\

{\bf Research Question:} How does the distribution of album sales vary across music genres for albums in the previous decade (released after 2015), and are high-sales outliers concentrated in certain genres?

\subsection{Part A: Data Cleaning and Preprocessing}
First, filter your dataset so that only the variables critical for your analysis remain. Then clean your data so that there is consistency in variable types, capitalization, and handle any missing or invalid values.

\subsection{Part B: Generate Three Visualizations}
Produce the following types of plots:
\begin{itemize}
    \item \textbf{Error Bar Plot:} Show the mean and variability (e.g., standard error or 95\% confidence intervals) of the numerical variable across each category.
    \item \textbf{Barcode Chart:} Also known as a strip plot or rug plot. Shows individual data points across categories.
    \item \textbf{Histogram:} Plot the distribution of the numerical variable, grouped by the categorical variable (using hue or facet).
\end{itemize}

\subsection{Part C: Evaluate and Justify Visualization}
For the dataset:
\begin{itemize}
    \item Discuss the advantages and disadvantages of each visualization type.
    \item Decide which visualization is best for the research question.
    \item Support your answer with evidence from the plots and reasoning based on dataset size, shape, or structure.
\end{itemize}
